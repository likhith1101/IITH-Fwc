\begin{abstract}
Through this manual, we will learn how to setting up the vaman-ESP as a Master and Arduino as a Slave using I2C protocol.
\end{abstract}
\subsection{Components}
\numberwithin{equation}{enumi}
\numberwithin{figure}{enumi}
%\numberwithin{table}{enumi}
\numberwithin{table}{section}


\begin{table}[!h]
\centering
\input{./vaman/vaman-esp/I2C/figs/components7.tex}
\caption{Components}
\label{table:i2c-components}
\end{table}


\subsection{Setting up the Master and Slave}
\begin{enumerate}[label=\thesection.\arabic*.,ref=\thesection.\theenumi]
\numberwithin{equation}{enumi}
\numberwithin{figure}{enumi}
\numberwithin{table}{enumi}

%
%\begin{figure}
%\centering
%\includegraphics[width=\columnwidth]{./vaman/esp32/I2C/figs/lcd.eps}
%\caption{lcd}
%\label{fig:lcd}
%\end{figure}
%
%
%%
%\item
%Connect  pin 16 (Led-) of the LCD to GND.  The LCD should glow.
%
%%
%\item
%Connect pin 3 of the LCD to GND.  This is required for contrast.
%


\item
Connect the vaman-ESP pins to Arduino pins as per Table \ref{Table:1 Arduino-ESP}.
\begin{table}[h]
\centering
\input{./vaman/vaman-esp/I2C/figs/table1.tex}
\caption{}
\label{Table:1 Arduino-ESP}
\end{table}

\item
Connect the vaman-ESP pins to LCD pins as per \ref{Table:1}..
%\begin{table}[h]
%\centering
%\input{./vaman/vaman-esp/I2C/figs/table2.tex}
%\caption{}
%\label{Table:2 ESP-LCD}
%\end{table}

\item The Vaman pin diagram is available in Fig. \ref{fig:vaman-pin_sheet}


\item
Configure Arduino Uno as a Slave using the following code.\\
\begin{lstlisting}
vaman/vaman-esp/I2C/codes/I2C_Sender_Arduino/src/main.cpp
\end{lstlisting}
\item
Now configure vaman-ESP as a Master using the following code.\\
\begin{lstlisting}
vaman/vaman-esp/I2C/codes/I2C_Reciever_ESP32/src/main.cpp
\end{lstlisting}

\end{enumerate}

%\end{enumerate}
%
%\section{Display Resistance on LCD}
%
%\begin{enumerate}[label=\thesection.\arabic*.,ref=\thesection.\theenumi]
%\numberwithin{equation}{enumi}
%\numberwithin{figure}{enumi}
%\numberwithin{table}{enumi}

%\begin{center}
%\begin{tabular}{ |p{3cm}|p{3cm}|}
 %\hline
 %\multicolumn{2}{|c|}{\textbf{ESP32 to LCD Connection}} \\
 %\hline
 %\textbf{ESP32 Pins} & \textbf{LCD Pins} \\
 %\hline
 %Pin D2 & Pin D7\\
 %\hline
 %Pin D3 & Pin D6\\
 %\hline
 %Pin D4 & Pin D5\\
 %\hline
 %Pin D5 & Pin D4\\
 %\hline
 %Pin 11 & Pin 6\\
 %\hline
 %Pin 12 & Pin 4\\
 %\hline
 %GND Pin & Pin 1 and Pin 5\\
 %\hline
 %5V  Pin & Pin 2\\
 %\hline
%\end{tabular}
%\end{center}

%
%%
%\item
%The Potentiometer to LCD connection are given below: 
%\begin{center}
%\begin{tabular}{ |p{3cm}|p{3cm}|}
 %\hline
 %\multicolumn{2}{|c|}{\textbf{Potentiometer to LCD Connection}} \\
 %\hline
 %\textbf{Potentiometer} & \textbf{LCD Pins} \\
 %\hline
 %Middle Pin & Pin 3\\
 %\hline
 %Pin 1 & Pin 2(Vcc Pin)\\
 %\hline
 %Pin 3 & Pin 1(GND Pin)\\
 %\hline
%\end{tabular}
%\end{center}
%
%
%
%\item
%The resistance is an analogue function,but the value displayed on LCD is digital function.So,we need to do analogue to digital conversion,ESP32 has built-in 10-bit analogue to digital converter.
%Two resistors are used, one of which is known resistor value and other is unknown resistor value.
%Take 2 resistors,one known resistor with resistance value(For e.g-1K,2K,10K),here we have taken 1K and another an unknown resistor whose value we are going to calculate.



