\def\mytitle{IMPLEMENTATION OF RANDOM GENERATOR USING\\ D FLIP-FLOPS IN ARDUINO IDE}
\def\mykeywords{}
\documentclass[10pt,a4paper]{article}
\usepackage[a4paper,outer=1.5cm,inner=1.5cm,top=1.75    cm,bottom=1.5cm]{geometry}
%  \twocolumn
\usepackage{graphicx}
\usepackage{amsfonts}
\usepackage{circuitikz}

\usepackage{tabularx}
\usepackage{tikz}
\usepackage{amsmath}
\usepackage[margin=1cm]{geometry}
\usetikzlibrary{shapes,arrows,chains,decorations.markings,intersections,calc}
\usepackage{lipsum}
\usetikzlibrary{positioning}
\usepackage{xcolor}
\usepackage{multirow}
\usepackage{listings}
\usepackage{float}
\usepackage{titlesec}
\usepackage{amsmath}
\usepackage[utf8]{inputenc}
\usepackage{algorithm2e}
\usepackage{karnaugh-map}                           
\usepackage{datetime}
\usepackage{lipsum}
\usepackage{amsmath}
\usepackage{textgreek}
\usepackage{circuitikz}
\usepackage{tikz}
\usetikzlibrary{calc}                         
\usetikzlibrary{circuits.logic.US}
\title{\mytitle}
 \author{MARIKUNDAM HARSHITHA\\marikundamdec@gmail.com\\FWC22120 IITH-Future Wireless Communications     Assignment-1}
\date{}
\sloppy
\lstset{                                          
language=C++,                           
basicstyle=\ttfamily\footnotesize,   
breaklines=true,                       
frame=lines
}

\begin{document}
\maketitle
\tableofcontents

\section{Problem}
(GATE2021-QP-EC)\\
Q.46 The propogation delay of the exclusive-OR(XOR) gate in the circuit in the figure is 3ns.The propogation delay of all the flip-flops is assumed to be zero.The clock(Clk) frequency provided to the circuit is 500MHz.\\
\begin{figure}[!h]
\begin{center}
\resizebox{0.5\columnwidth}{!}{
\begin{tikzpicture}
\ctikzset{                                   
logic ports=ieee,                   
logic ports/scale=0.5               
}                                    
\draw(-1.3,0)node[xor port,anchor=out](x) {};         
\tikzstyle{dff}=[rectangle,draw,minimum height=7em,text width=7em,inner sep=3em]                                       
\node[dff] (dff2) {D2};                             
\node[dff, right=2cm of dff2] (dff1) {D1};           
\node[dff, right=2cm of dff1] (dff0) {D0};        
%Connecting flip-flops together                    
\draw (dff2.out) -- ++(2,0) node[above]{};        
\draw (dff1.out) -- ++(2,0) node[above] {};         
\draw (dff0.out) -- ++(2,0) node[above]{};          
\draw(dff2.out) -| (2.3,1.5) node[above]{$Q2$};        
\draw(dff1.out) -|(6.8,1.2) node[above]{$Q1$};         
\draw(dff0.out) -|(12.4,2) node[above]{$Q0$};          
\draw(x.in 2) -|(-3,2)to[short] (12.4,2);              
\draw(x.in 1)-|(-2.5,1.5)to[short](2.3,1.5);          
\draw(-2,-2) node[above]{$Clk$} --(6,-2);            
\draw(6,-2) node[above]{} --(9.1,-2);                 
\draw(9.1,-2)--(9.1,-1.2) node[above]{};                 
\draw(8.9,-1.23)--(9.1,-1)--(9.3,-1.23);               
\draw(4.5,-2)--(4.5,-1.2) node[above]{};               
\draw(4.3,-1.23)--(4.5,-1)--(4.7,-1.23);            
\draw(0,-2)--(0,-1.2) node[above]{};               
\draw(-0.2,-1.23)--(0,-1)--(0.2,-1.23);
\end{tikzpicture}

}
\end{center}
	\caption{Circuit}
\label{fig:circuit}
\end{figure}

Starting from the initial value of the flip-flop outputs $Q2Q1Q0 =111$ with $D2=1$,the minimum number of triggering clock edges after which the flip-flop outputs $Q2Q1Q0$ becomes 1 0 0\emph{(in integer)} is \line(1,0){12.5}

\section{Introduction}
A random number generator using D flip-flops is a simple digital circuit that generates a sequence of random binary numbers.To implement this type of random number generator, we use a series of D flip-flops connected in a feedback loop. The output of each flip-flop is fed back into the input of the next flip-flop,creating a circuit that generated a sequence of random binary values.\\ \\
The feedback loop creates a delay in the circuit,which causes the circuit to exhibit unpredictable behavior.This unpredictable behavior results in a sequence of random binary values. The length of the delay can be adjusted to control the randomness of the output.

\section{Components}
\begin{table}[!h]
\centering
%%%%%%%%%%%%%%%%%%%%%%%%%%%%%%%%%%%%%%%%%%%%%%%%%%%%%%%%%%%%%%%%%%%%%%
%%                                                                  %%
%%  This is the header of a LaTeX2e file exported from Gnumeric.    %%
%%                                                                  %%
%%  This file can be compiled as it stands or included in another   %%
%%  LaTeX document. The table is based on the longtable package so  %%
%%  the longtable options (headers, footers...) can be set in the   %%
%%  preamble section below (see PRAMBLE).                           %%
%%                                                                  %%
%%  To include the file in another, the following two lines must be %%
%%  in the including file:                                          %%
%%        \def\inputGnumericTable{}                                 %%
%%  at the beginning of the file and:                               %%
%%        \input{name-of-this-file.tex}                             %%
%%  where the table is to be placed. Note also that the including   %%
%%  file must use the following packages for the table to be        %%
%%  rendered correctly:                                             %%
%%    \usepackage[latin1]{inputenc}                                 %%
%%    \usepackage{color}                                            %%
%%    \usepackage{array}                                            %%
%%    \usepackage{longtable}                                        %%
%%    \usepackage{calc}                                             %%
%%    \usepackage{multirow}                                         %%
%%    \usepackage{hhline}                                           %%
%%    \usepackage{ifthen}                                           %%
%%  optionally (for landscape tables embedded in another document): %%
%%    \usepackage{lscape}                                           %%
%%                                                                  %%
%%%%%%%%%%%%%%%%%%%%%%%%%%%%%%%%%%%%%%%%%%%%%%%%%%%%%%%%%%%%%%%%%%%%%%



%%  This section checks if we are begin input into another file or  %%
%%  the file will be compiled alone. First use a macro taken from   %%
%%  the TeXbook ex 7.7 (suggestion of Han-Wen Nienhuys).            %%
\def\ifundefined#1{\expandafter\ifx\csname#1\endcsname\relax}


%%  Check for the \def token for inputed files. If it is not        %%
%%  defined, the file will be processed as a standalone and the     %%
%%  preamble will be used.                                          %%
\ifundefined{inputGnumericTable}

%%  We must be able to close or not the document at the end.        %%
	\def\gnumericTableEnd{\end{document}}


%%%%%%%%%%%%%%%%%%%%%%%%%%%%%%%%%%%%%%%%%%%%%%%%%%%%%%%%%%%%%%%%%%%%%%
%%                                                                  %%
%%  This is the PREAMBLE. Change these values to get the right      %%
%%  paper size and other niceties.                                  %%
%%                                                                  %%
%%%%%%%%%%%%%%%%%%%%%%%%%%%%%%%%%%%%%%%%%%%%%%%%%%%%%%%%%%%%%%%%%%%%%%

	\documentclass[12pt%
			  %,landscape%
                    ]{report}
       \usepackage[latin1]{inputenc}
       \usepackage{fullpage}
       \usepackage{color}
       \usepackage{array}
       \usepackage{longtable}
       \usepackage{calc}
       \usepackage{multirow}
       \usepackage{hhline}
       \usepackage{ifthen}

	\begin{document}


%%  End of the preamble for the standalone. The next section is for %%
%%  documents which are included into other LaTeX2e files.          %%
\else

%%  We are not a stand alone document. For a regular table, we will %%
%%  have no preamble and only define the closing to mean nothing.   %%
    \def\gnumericTableEnd{}

%%  If we want landscape mode in an embedded document, comment out  %%
%%  the line above and uncomment the two below. The table will      %%
%%  begin on a new page and run in landscape mode.                  %%
%       \def\gnumericTableEnd{\end{landscape}}
%       \begin{landscape}


%%  End of the else clause for this file being \input.              %%
\fi

%%%%%%%%%%%%%%%%%%%%%%%%%%%%%%%%%%%%%%%%%%%%%%%%%%%%%%%%%%%%%%%%%%%%%%
%%                                                                  %%
%%  The rest is the gnumeric table, except for the closing          %%
%%  statement. Changes below will alter the table's appearance.     %%
%%                                                                  %%
%%%%%%%%%%%%%%%%%%%%%%%%%%%%%%%%%%%%%%%%%%%%%%%%%%%%%%%%%%%%%%%%%%%%%%

\providecommand{\gnumericmathit}[1]{#1} 
%%  Uncomment the next line if you would like your numbers to be in %%
%%  italics if they are italizised in the gnumeric table.           %%
%\renewcommand{\gnumericmathit}[1]{\mathit{#1}}
\providecommand{\gnumericPB}[1]%
{\let\gnumericTemp=\\#1\let\\=\gnumericTemp\hspace{0pt}}
 \ifundefined{gnumericTableWidthDefined}
        \newlength{\gnumericTableWidth}
        \newlength{\gnumericTableWidthComplete}
        \newlength{\gnumericMultiRowLength}
        \global\def\gnumericTableWidthDefined{}
 \fi
%% The following setting protects this code from babel shorthands.  %%
 \ifthenelse{\isundefined{\languageshorthands}}{}{\languageshorthands{english}}
%%  The default table format retains the relative column widths of  %%
%%  gnumeric. They can easily be changed to c, r or l. In that case %%
%%  you may want to comment out the next line and uncomment the one %%
%%  thereafter                                                      %%
\providecommand\gnumbox{\makebox[0pt]}
%%\providecommand\gnumbox[1][]{\makebox}

%% to adjust positions in multirow situations                       %%
\setlength{\bigstrutjot}{\jot}
\setlength{\extrarowheight}{\doublerulesep}

%%  The \setlongtables command keeps column widths the same across  %%
%%  pages. Simply comment out next line for varying column widths.  %%
\setlongtables

\setlength\gnumericTableWidth{%
	45pt+%
	30pt+%
	52pt+%
	60pt+%
0pt}
\def\gumericNumCols{4}
\setlength\gnumericTableWidthComplete{\gnumericTableWidth+%
         \tabcolsep*\gumericNumCols*2+\arrayrulewidth*\gumericNumCols}
\ifthenelse{\lengthtest{\gnumericTableWidthComplete > \linewidth}}%
         {\def\gnumericScale{\ratio{\linewidth-%
                        \tabcolsep*\gumericNumCols*2-%
                        \arrayrulewidth*\gumericNumCols}%
{\gnumericTableWidth}}}%
{\def\gnumericScale{1}}

%%%%%%%%%%%%%%%%%%%%%%%%%%%%%%%%%%%%%%%%%%%%%%%%%%%%%%%%%%%%%%%%%%%%%%
%%                                                                  %%
%% The following are the widths of the various columns. We are      %%
%% defining them here because then they are easier to change.       %%
%% Depending on the cell formats we may use them more than once.    %%
%%                                                                  %%
%%%%%%%%%%%%%%%%%%%%%%%%%%%%%%%%%%%%%%%%%%%%%%%%%%%%%%%%%%%%%%%%%%%%%%

\ifthenelse{\isundefined{\gnumericColA}}{\newlength{\gnumericColA}}{}\settowidth{\gnumericColA}{\begin{tabular}{@{}p{45pt*\gnumericScale}@{}}x\end{tabular}}
\ifthenelse{\isundefined{\gnumericColB}}{\newlength{\gnumericColB}}{}\settowidth{\gnumericColB}{\begin{tabular}{@{}p{30pt*\gnumericScale}@{}}x\end{tabular}}
\ifthenelse{\isundefined{\gnumericColC}}{\newlength{\gnumericColC}}{}\settowidth{\gnumericColC}{\begin{tabular}{@{}p{52pt*\gnumericScale}@{}}x\end{tabular}}
\ifthenelse{\isundefined{\gnumericColD}}{\newlength{\gnumericColD}}{}\settowidth{\gnumericColD}{\begin{tabular}{@{}p{60pt*\gnumericScale}@{}}x\end{tabular}}

\begin{tabular}[c]{%
	b{\gnumericColA}%
	b{\gnumericColB}%
	b{\gnumericColC}%
	b{\gnumericColD}%
	}

%%%%%%%%%%%%%%%%%%%%%%%%%%%%%%%%%%%%%%%%%%%%%%%%%%%%%%%%%%%%%%%%%%%%%%
%%  The longtable options. (Caption, headers... see Goosens, p.124) %%
%	\caption{The Table Caption.}             \\	%
% \hline	% Across the top of the table.
%%  The rest of these options are table rows which are placed on    %%
%%  the first, last or every page. Use \multicolumn if you want.    %%

%%  Header for the first page.                                      %%
%	\multicolumn{4}{c}{The First Header} \\ \hline 
%	\multicolumn{1}{c}{colTag}	%Column 1
%	&\multicolumn{1}{c}{colTag}	%Column 2
%	&\multicolumn{1}{c}{colTag}	%Column 3
%	&\multicolumn{1}{c}{colTag}	\\ \hline %Last column
%	\endfirsthead

%%  The running header definition.                                  %%
%	\hline
%	\multicolumn{4}{l}{\ldots\small\slshape continued} \\ \hline
%	\multicolumn{1}{c}{colTag}	%Column 1
%	&\multicolumn{1}{c}{colTag}	%Column 2
%	&\multicolumn{1}{c}{colTag}	%Column 3
%	&\multicolumn{1}{c}{colTag}	\\ \hline %Last column
%	\endhead

%%  The running footer definition.                                  %%
%	\hline
%	\multicolumn{4}{r}{\small\slshape continued\ldots} \\
%	\endfoot

%%  The ending footer definition.                                   %%
%	\multicolumn{4}{c}{That's all folks} \\ \hline 
%	\endlastfoot
%%%%%%%%%%%%%%%%%%%%%%%%%%%%%%%%%%%%%%%%%%%%%%%%%%%%%%%%%%%%%%%%%%%%%%

\hhline{|-|-|-|-}
	 \multicolumn{1}{|p{\gnumericColA}|}%
	{\gnumericPB{\raggedright}\textbf{ESP32}}
	&\multicolumn{1}{p{\gnumericColB}|}%
	{\gnumericPB{\raggedright}\textbf{LCD Pins}}
	&\multicolumn{1}{p{\gnumericColC}|}%
	{\gnumericPB{\raggedright}\textbf{LCD Pin Label}}
	&\multicolumn{1}{p{\gnumericColD}|}%
	{\gnumericPB{\raggedright}\textbf{LCD Pin Description}}
\\
\hhline{|----|}
	 \multicolumn{1}{|p{\gnumericColA}|}%
	{\gnumericPB{\raggedright}GND}
	&\multicolumn{1}{p{\gnumericColB}|}%
	{\gnumericPB{\raggedright}1}
	&\multicolumn{1}{p{\gnumericColC}|}%
	{\gnumericPB{\raggedright}GND }
	&\multicolumn{1}{p{\gnumericColD}|}%
	{\setlength{\gnumericMultiRowLength}{0pt}%
	 \addtolength{\gnumericMultiRowLength}{\gnumericColD}%
	 \multirow{2}[1]{\gnumericMultiRowLength}{%
	 }}
\\
\hhline{|---|~}
	 \multicolumn{1}{|p{\gnumericColA}|}%
	{\gnumericPB{\raggedright}5V}
	&\multicolumn{1}{p{\gnumericColB}|}%
	{\gnumericPB{\raggedright}2}
	&\multicolumn{1}{p{\gnumericColC}|}%
	{\gnumericPB{\raggedright}Vcc}
	&\multicolumn{1}{p{\gnumericColD}|}%
	{}
\\
\hhline{|----|}
	 \multicolumn{1}{|p{\gnumericColA}|}%
	{\gnumericPB{\raggedright}GND}
	&\multicolumn{1}{p{\gnumericColB}|}%
	{\gnumericPB{\raggedright}3}
	&\multicolumn{1}{p{\gnumericColC}|}%
	{\gnumericPB{\raggedright}Vee}
	&\multicolumn{1}{p{\gnumericColD}|}%
	{\gnumericPB{\raggedright}Contrast}
\\
\hhline{|----|}
	 \multicolumn{1}{|p{\gnumericColA}|}%
	{\gnumericPB{\raggedright}GPIO 19}
	&\multicolumn{1}{p{\gnumericColB}|}%
	{\gnumericPB{\raggedright}4}
	&\multicolumn{1}{p{\gnumericColC}|}%
	{\gnumericPB{\raggedright}RS}
	&\multicolumn{1}{p{\gnumericColD}|}%
	{\gnumericPB{\raggedright}Register Select}
\\
\hhline{|----|}
	 \multicolumn{1}{|p{\gnumericColA}|}%
	{\gnumericPB{\raggedright}GND}
	&\multicolumn{1}{p{\gnumericColB}|}%
	{\gnumericPB{\raggedright}5}
	&\multicolumn{1}{p{\gnumericColC}|}%
	{\gnumericPB{\raggedright}R/W}
	&\multicolumn{1}{p{\gnumericColD}|}%
	{\gnumericPB{\raggedright}Read/Write}
\\
\hhline{|----|}
	 \multicolumn{1}{|p{\gnumericColA}|}%
	{\gnumericPB{\raggedright}GPIO 23}
	&\multicolumn{1}{p{\gnumericColB}|}%
	{\gnumericPB{\raggedright}6}
	&\multicolumn{1}{p{\gnumericColC}|}%
	{\gnumericPB{\raggedright}EN}
	&\multicolumn{1}{p{\gnumericColD}|}%
	{\gnumericPB{\raggedright}Enable}
\\
\hhline{|----|}
	 \multicolumn{1}{|p{\gnumericColA}|}%
	{\gnumericPB{\raggedright}GPIO 18}
	&\multicolumn{1}{p{\gnumericColB}|}%
	{\gnumericPB{\raggedright}11}
	&\multicolumn{1}{p{\gnumericColC}|}%
	{\gnumericPB{\raggedright}DB4}
	&\multicolumn{1}{p{\gnumericColD}|}%
	{\gnumericPB{\raggedright}Serial Connection}
\\
\hhline{|----|}
	 \multicolumn{1}{|p{\gnumericColA}|}%
	{\gnumericPB{\raggedright}GPIO 17}
	&\multicolumn{1}{p{\gnumericColB}|}%
	{\gnumericPB{\raggedright}12}
	&\multicolumn{1}{p{\gnumericColC}|}%
	{\gnumericPB{\raggedright}DB5}
	&\multicolumn{1}{p{\gnumericColD}|}%
	{\gnumericPB{\raggedright}Serial Connection}
\\
\hhline{|----|}
	 \multicolumn{1}{|p{\gnumericColA}|}%
	{\gnumericPB{\raggedright}GPIO 16}
	&\multicolumn{1}{p{\gnumericColB}|}%
	{\gnumericPB{\raggedright}13}
	&\multicolumn{1}{p{\gnumericColC}|}%
	{\gnumericPB{\raggedright}DB6}
	&\multicolumn{1}{p{\gnumericColD}|}%
	{\gnumericPB{\raggedright}Serial Connection}
\\
\hhline{|----|}
	 \multicolumn{1}{|p{\gnumericColA}|}%
	{\gnumericPB{\raggedright}GPIO 15}
	&\multicolumn{1}{p{\gnumericColB}|}%
	{\gnumericPB{\raggedright}14}
	&\multicolumn{1}{p{\gnumericColC}|}%
	{\gnumericPB{\raggedright}DB7}
	&\multicolumn{1}{p{\gnumericColD}|}%
	{\gnumericPB{\raggedright}Serial Connection}
\\
\hhline{|----|}
	 \multicolumn{1}{|p{\gnumericColA}|}%
	{\gnumericPB{\raggedright}5V}
	&\multicolumn{1}{p{\gnumericColB}|}%
	{\gnumericPB{\raggedright}15}
	&\multicolumn{1}{p{\gnumericColC}|}%
	{\gnumericPB{\raggedright}LED+}
	&\multicolumn{1}{p{\gnumericColD}|}%
	{\gnumericPB{\raggedright}Backlight}
\\
\hhline{|----|}
	 \multicolumn{1}{|p{\gnumericColA}|}%
	{\gnumericPB{\raggedright}GND}
	&\multicolumn{1}{p{\gnumericColB}|}%
	{\gnumericPB{\raggedright}16}
	&\multicolumn{1}{p{\gnumericColC}|}%
	{\gnumericPB{\raggedright}LED-}
	&\multicolumn{1}{p{\gnumericColD}|}%
	{\gnumericPB{\raggedright}Backlight}
\\
\hhline{|-|-|-|-|}
\end{tabular}

\ifthenelse{\isundefined{\languageshorthands}}{}{\languageshorthands{\languagename}}
\gnumericTableEnd

\caption{Components}
\label{table:components}
\end{table}
\subsection{Arduino} 
The Arduino Uno has some ground pins,analog input pins A0-A3 and digital pins D1-D13 that can be used for both input as well as output.It also has two power pins that can generate 3.3V and 5V.Inthe following exercises, we use digital pins,GND and 5V .
\subsection{Seven Segment Display}
The seven segment display has eight pins, \emph{a,b,c,d,e,f,g} and \emph{dot} that take an active LOW input,i.e. the LED will glow only if the input is connected to ground.Each of these pins is connected to an LED segment.The \emph{dot} pin is reserved for the LED.
\section{Implementation}
A 7474 IC which  has 14 pins and can store two seperate binary values.So we consider two IC's since we have three values  and connect the  D inputs of each flip-flop to the input signals of 7447 IC . Later interface 7447 IC to seven segment display for the output. The CLK input is used to trigger the flip-flop,and the Q output is used to read the stored value.When a positive edge is detected on the CLK input,the current value on the D input is stored in the flip-flop. The boolean expression of the D flip-flop is $Q(t+1) = D$
\subsection{Truth table}
\begin{table}[!h]
\centering
\captionof{table}{Table2}
\label{table:2}
\begin{tabular}{|p{3cm}|p{1cm}|p{1cm}|p{1cm}|p{1cm}|p{1cm}|p{1cm}|p{1cm}|}                                           
	\hline                                                
	\multicolumn{8}{|c|}{7447 - Display}\\                                                                    
	\hline                                                
	7447& $\bar{a}$ & $\bar{b}$ & $\bar{c}$ & $\bar{d}$ & $\bar{e}$ & $\bar{f}$ & $\bar{g}$\\                                                                    
	\hline                                                
	Display& a& b& c& d& e& f& g\\                                                                            
	\hline                                        
\end{tabular}
   
\caption{Truth Table}
\label{table:truth_table}
\end{table}
\subsection{K-map}
Since $Q'= D$,we find the k-maps for D as outputs\\
\begin{figure}[!h]                                
\begin{center}                                 
\resizebox{0.5\columnwidth}{!}{
\begin{karnaugh-map}[4][2][1][$Q2$ $Q1$][$Q0$]    
\minterms{2,3,4,5}                    
\maxterms{0,1,6,7}                
\implicant{3}{2}              
\implicant{4}{5}          
\end{karnaugh-map}

}                                                
\end{center}                                     
\caption{For D2}                                       
\label{fig:for_D2}                             
\end{figure}
%
\begin{figure}[!h]                              
\begin{center}                                 
\resizebox{0.5\columnwidth}{!}{
\begin{karnaugh-map}[4][2][1][$Q2$ $Q1$][$Q0$]  
\minterms{2,3,7,6}            
\maxterms{0,1,4,5}         
\implicant{3}{6}              
\end{karnaugh-map}

}                                               
\end{center}                                   
\caption{For D1}                                     
\label{fig:for_D1}                             
\end{figure}
%
\begin{figure}[!h]                             
\begin{center}                                
\resizebox{0.5\columnwidth}{!}{
\begin{karnaugh-map}[4][2][1][$Q2$ $Q1$][$Q0$]    
\minterms{1,3,5,7}                   
\maxterms{0,2,4,6}            
\implicant{1}{7}            
\end{karnaugh-map}

}                                              
\end{center}                              
\caption{For D0}                                
\label{fig:for_D1}                            
\end{figure}                                      
%
\subsection{Boolean Equation}
By solving the K-maps above we obtain as follows :
\begin{align}
	D2 &= \overline{Q2}Q0 + \overline{Q0}Q2 \\
	D1 &= Q2 \\
	D0 &= Q1 
\end{align}
\section{Hardware}
\begin{enumerate}
\item Make the connections between the seven segment display and the 7447 IC as shown in Table3
\begin{table}[!h]                                
\centering
%%%%%%%%%%%%%%%%%%%%%%%%%%%%%%%%%%%%%%%%%%%%%%%%%%%%%%%%%%%%%%%%%%%%%%
%%                                                                  %%
%%  This is the header of a LaTeX2e file exported from Gnumeric.    %%
%%                                                                  %%
%%  This file can be compiled as it stands or included in another   %%
%%  LaTeX document. The table is based on the longtable package so  %%
%%  the longtable options (headers, footers...) can be set in the   %%
%%  preamble section below (see PRAMBLE).                           %%
%%                                                                  %%
%%  To include the file in another, the following two lines must be %%
%%  in the including file:                                          %%
%%        \def\inputGnumericTable{}                                 %%
%%  at the beginning of the file and:                               %%
%%        \input{name-of-this-file.tex}                             %%
%%  where the table is to be placed. Note also that the including   %%
%%  file must use the following packages for the table to be        %%
%%  rendered correctly:                                             %%
%%    \usepackage[latin1]{inputenc}                                 %%
%%    \usepackage{color}                                            %%
%%    \usepackage{array}                                            %%
%%    \usepackage{longtable}                                        %%
%%    \usepackage{calc}                                             %%
%%    \usepackage{multirow}                                         %%
%%    \usepackage{hhline}                                           %%
%%    \usepackage{ifthen}                                           %%
%%  optionally (for landscape tables embedded in another document): %%
%%    \usepackage{lscape}                                           %%
%%                                                                  %%
%%%%%%%%%%%%%%%%%%%%%%%%%%%%%%%%%%%%%%%%%%%%%%%%%%%%%%%%%%%%%%%%%%%%%%



%%  This section checks if we are begin input into another file or  %%
%%  the file will be compiled alone. First use a macro taken from   %%
%%  the TeXbook ex 7.7 (suggestion of Han-Wen Nienhuys).            %%
\def\ifundefined#1{\expandafter\ifx\csname#1\endcsname\relax}


%%  Check for the \def token for inputed files. If it is not        %%
%%  defined, the file will be processed as a standalone and the     %%
%%  preamble will be used.                                          %%
\ifundefined{inputGnumericTable}

%%  We must be able to close or not the document at the end.        %%
	\def\gnumericTableEnd{\end{document}}


%%%%%%%%%%%%%%%%%%%%%%%%%%%%%%%%%%%%%%%%%%%%%%%%%%%%%%%%%%%%%%%%%%%%%%
%%                                                                  %%
%%  This is the PREAMBLE. Change these values to get the right      %%
%%  paper size and other niceties.                                  %%
%%                                                                  %%
%%%%%%%%%%%%%%%%%%%%%%%%%%%%%%%%%%%%%%%%%%%%%%%%%%%%%%%%%%%%%%%%%%%%%%

	\documentclass[12pt%
			  %,landscape%
                    ]{report}
       \usepackage[latin1]{inputenc}
       \usepackage{fullpage}
       \usepackage{color}
       \usepackage{array}
       \usepackage{longtable}
       \usepackage{calc}
       \usepackage{multirow}
       \usepackage{hhline}
       \usepackage{ifthen}

	\begin{document}


%%  End of the preamble for the standalone. The next section is for %%
%%  documents which are included into other LaTeX2e files.          %%
\else

%%  We are not a stand alone document. For a regular table, we will %%
%%  have no preamble and only define the closing to mean nothing.   %%
    \def\gnumericTableEnd{}

%%  If we want landscape mode in an embedded document, comment out  %%
%%  the line above and uncomment the two below. The table will      %%
%%  begin on a new page and run in landscape mode.                  %%
%       \def\gnumericTableEnd{\end{landscape}}
%       \begin{landscape}


%%  End of the else clause for this file being \input.              %%
\fi

%%%%%%%%%%%%%%%%%%%%%%%%%%%%%%%%%%%%%%%%%%%%%%%%%%%%%%%%%%%%%%%%%%%%%%
%%                                                                  %%
%%  The rest is the gnumeric table, except for the closing          %%
%%  statement. Changes below will alter the table's appearance.     %%
%%                                                                  %%
%%%%%%%%%%%%%%%%%%%%%%%%%%%%%%%%%%%%%%%%%%%%%%%%%%%%%%%%%%%%%%%%%%%%%%

\providecommand{\gnumericmathit}[1]{#1} 
%%  Uncomment the next line if you would like your numbers to be in %%
%%  italics if they are italizised in the gnumeric table.           %%
%\renewcommand{\gnumericmathit}[1]{\mathit{#1}}
\providecommand{\gnumericPB}[1]%
{\let\gnumericTemp=\\#1\let\\=\gnumericTemp\hspace{0pt}}
 \ifundefined{gnumericTableWidthDefined}
        \newlength{\gnumericTableWidth}
        \newlength{\gnumericTableWidthComplete}
        \newlength{\gnumericMultiRowLength}
        \global\def\gnumericTableWidthDefined{}
 \fi
%% The following setting protects this code from babel shorthands.  %%
 \ifthenelse{\isundefined{\languageshorthands}}{}{\languageshorthands{english}}
%%  The default table format retains the relative column widths of  %%
%%  gnumeric. They can easily be changed to c, r or l. In that case %%
%%  you may want to comment out the next line and uncomment the one %%
%%  thereafter                                                      %%
\providecommand\gnumbox{\makebox[0pt]}
%%\providecommand\gnumbox[1][]{\makebox}

%% to adjust positions in multirow situations                       %%
\setlength{\bigstrutjot}{\jot}
\setlength{\extrarowheight}{\doublerulesep}

%%  The \setlongtables command keeps column widths the same across  %%
%%  pages. Simply comment out next line for varying column widths.  %%
\setlongtables

\setlength\gnumericTableWidth{%
	125pt+%
	121pt+%
0pt}
\def\gumericNumCols{2}
\setlength\gnumericTableWidthComplete{\gnumericTableWidth+%
         \tabcolsep*\gumericNumCols*2+\arrayrulewidth*\gumericNumCols}
\ifthenelse{\lengthtest{\gnumericTableWidthComplete > \linewidth}}%
         {\def\gnumericScale{\ratio{\linewidth-%
                        \tabcolsep*\gumericNumCols*2-%
                        \arrayrulewidth*\gumericNumCols}%
{\gnumericTableWidth}}}%
{\def\gnumericScale{1}}

%%%%%%%%%%%%%%%%%%%%%%%%%%%%%%%%%%%%%%%%%%%%%%%%%%%%%%%%%%%%%%%%%%%%%%
%%                                                                  %%
%% The following are the widths of the various columns. We are      %%
%% defining them here because then they are easier to change.       %%
%% Depending on the cell formats we may use them more than once.    %%
%%                                                                  %%
%%%%%%%%%%%%%%%%%%%%%%%%%%%%%%%%%%%%%%%%%%%%%%%%%%%%%%%%%%%%%%%%%%%%%%

\ifthenelse{\isundefined{\gnumericColA}}{\newlength{\gnumericColA}}{}\settowidth{\gnumericColA}{\begin{tabular}{@{}p{125pt*\gnumericScale}@{}}x\end{tabular}}
\ifthenelse{\isundefined{\gnumericColB}}{\newlength{\gnumericColB}}{}\settowidth{\gnumericColB}{\begin{tabular}{@{}p{121pt*\gnumericScale}@{}}x\end{tabular}}

\begin{longtable}[c]{%
	b{\gnumericColA}%
	b{\gnumericColB}%
	}

%%%%%%%%%%%%%%%%%%%%%%%%%%%%%%%%%%%%%%%%%%%%%%%%%%%%%%%%%%%%%%%%%%%%%%
%%  The longtable options. (Caption, headers... see Goosens, p.124) %%
%	\caption{The Table Caption.}             \\	%
% \hline	% Across the top of the table.
%%  The rest of these options are table rows which are placed on    %%
%%  the first, last or every page. Use \multicolumn if you want.    %%

%%  Header for the first page.                                      %%
%	\multicolumn{2}{c}{The First Header} \\ \hline 
%	\multicolumn{1}{c}{colTag}	%Column 1
%	&\multicolumn{1}{c}{colTag}	\\ \hline %Last column
%	\endfirsthead

%%  The running header definition.                                  %%
%	\hline
%	\multicolumn{2}{l}{\ldots\small\slshape continued} \\ \hline
%	\multicolumn{1}{c}{colTag}	%Column 1
%	&\multicolumn{1}{c}{colTag}	\\ \hline %Last column
%	\endhead

%%  The running footer definition.                                  %%
%	\hline
%	\multicolumn{2}{r}{\small\slshape continued\ldots} \\
%	\endfoot

%%  The ending footer definition.                                   %%
%	\multicolumn{2}{c}{That's all folks} \\ \hline 
%	\endlastfoot
%%%%%%%%%%%%%%%%%%%%%%%%%%%%%%%%%%%%%%%%%%%%%%%%%%%%%%%%%%%%%%%%%%%%%%

\hhline{|-|-}
	 \multicolumn{1}{|p{\gnumericColA}|}%
	{\gnumericPB{\raggedright}\gnumbox[l]{\textbf{Vaman Board ESP 32}}}
	&\multicolumn{1}{p{\gnumericColB}|}%
	{\gnumericPB{\raggedright}\gnumbox[l]{\textbf{Motor Driver Unit}}}
\\
\hhline{|--|}
	 \multicolumn{1}{|p{\gnumericColA}|}%
	{\gnumericPB{\raggedright}\gnumbox[l]{Pin 16}}
	&\multicolumn{1}{p{\gnumericColB}|}%
	{\gnumericPB{\raggedright}\gnumbox[l]{Right Motor Input 1}}
\\
\hhline{|--|}
	 \multicolumn{1}{|p{\gnumericColA}|}%
	{\gnumericPB{\raggedright}\gnumbox[l]{Pin 17}}
	&\multicolumn{1}{p{\gnumericColB}|}%
	{\gnumericPB{\raggedright}\gnumbox[l]{Right Motor Input 2}}
\\
\hhline{|--|}
	 \multicolumn{1}{|p{\gnumericColA}|}%
	{\gnumericPB{\raggedright}\gnumbox[l]{Pin 18}}
	&\multicolumn{1}{p{\gnumericColB}|}%
	{\gnumericPB{\raggedright}\gnumbox[l]{Left Motor Input 1}}
\\
\hhline{|--|}
	 \multicolumn{1}{|p{\gnumericColA}|}%
	{\gnumericPB{\raggedright}\gnumbox[l]{Pin 19}}
	&\multicolumn{1}{p{\gnumericColB}|}%
	{\gnumericPB{\raggedright}\gnumbox[l]{Left Motor Input 2}}
\\
\hhline{|--|}
	 \multicolumn{1}{|p{\gnumericColA}|}%
	{\gnumericPB{\raggedright}\gnumbox[l]{5v}}
	&\multicolumn{1}{p{\gnumericColB}|}%
	{\gnumericPB{\raggedright}\gnumbox[l]{VCC}}
\\
\hhline{|--|}
	 \multicolumn{1}{|p{\gnumericColA}|}%
	{\gnumericPB{\raggedright}\gnumbox[l]{GND}}
	&\multicolumn{1}{p{\gnumericColB}|}%
	{\gnumericPB{\raggedright}\gnumbox[l]{GND}}
\\
\hhline{|-|-|}
\end{longtable}

\ifthenelse{\isundefined{\languageshorthands}}{}{\languageshorthands{\languagename}}
\gnumericTableEnd
   
\caption{7447}                               
\label{table:7447}                       
\end{table}
\item Connect the Arduino,7447 and the two 7474 ICs according to Table4
\begin{table}[!h]                                 
\centering	
\begin{tabular}{|c|c|c|c|c|c|c|c|c|c|c|c|c|}      
\hline                              
\multirow{2}{*}{} & \multicolumn{3}{|c|}{INPUT} & \multicolumn{3}{|c|}{OUTPUT} & \multicolumn{2}{|c|}{\multirow{2}{*}{CLOCK}} & \multicolumn{4}{|c|}{\multirow{3}{*}{5V}} \\      
\cline{2-7}     
& Q0 & Q1 & Q2 & Q0' & Q1' & Q2' & \multicolumn{2}{|c|}{\multirow{2}{*}{}} & \multicolumn{4}{|c|}{} \\        
\hline          
Arduino & D6 & D7 & D8 & D2 & D3 & D4 & \multicolumn{2}{|c|}{D13} & \multicolumn{4}{|c|}{\multirow{3}{*}{}}\\                                   
\hline                             
7474 & 5 & 9 &  & 2 & 12 &  & CLK1 & CLK2 & 1 & 4 & 10 & 13 \\                   
\hline                     
7474 & & & 5 & & & 2 & CLK1 & CLK2 & 1 & 4  & 10 & 13 \\                       
\hline                         
7447 & \multicolumn{3}{|c|}{} & 7 & 1 & 2 & & & \multicolumn{4}{|c|}{16} \\              
\hline
\end{tabular}
 
\caption{Connections}                                   
\label{table:connections}                       
\end{table}
\item Make the other D input pins of 7474 grounded and supply  5V and GND from the arduino as well.
\item When the clock edge is trigerred we observe display of random numbers.
\end{enumerate}
\section{Software}
Now write the following code and upload in arduino to see the results.
\lstinputlisting{code6.cpp}
\end{document}
