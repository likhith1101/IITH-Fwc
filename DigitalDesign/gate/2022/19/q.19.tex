\documentclass[12pt]{article}
\usepackage{graphicx}
\usepackage{gensymb}
\usepackage[none]{hyphenat}
\usepackage{graphicx}
\usepackage{listings}
\usepackage[english]{babel}
\usepackage{graphicx}
\usepackage{caption}
\usepackage{hyperref}
\usepackage{booktabs}
\usepackage{array}
\usepackage{amsmath}
\usepackage{listings}
\usepackage{multirow}
\usepackage{blindtext}
\usepackage{capt-of}
\usepackage{circuitikz}
\usepackage{./karnaugh-map}
\usetikzlibrary{shapes.geometric}
\title{Implementation of 4x1 mux in Arduino using ICs}
\date{February 2023}
\lstset{
	frame=single'
	breaklines=true
}
\newcommand{\mydet}[1]{\ensuremath{\begin{vmatrix}#1\end{vmatrix}}}
\providecommand{\brak}[1]{\ensuremath{\left(#1\right)}}
\providecommand{\norm}[1]{\left\lvert#1\right\rVert}
\newcommand{\solution}{\noindent \textbf{Solution: }}
\newcommand{\myvec}[1]{\ensuremath{\begin{pmatrix}#1\end{pmatrix}}}
\let\vec\mathbf

\begin{document}
\maketitle
\tableofcontents

	 \section{Problem}
	 (GATE EC-2022)\\

Q.19. Consider the 2-bit multiplexer(MUX) shown in the figure.For output to be the XOR of R and S,the values for $ W,X,Y$ and $Z$ are ?\newline
\begin{figure}[h]
\begin{tikzpicture}
\ctikzset{                                   
logic ports=ieee,                   
logic ports/scale=0.5               
}                                    
\draw(-1.3,0)node[xor port,anchor=out](x) {};         
\tikzstyle{dff}=[rectangle,draw,minimum height=7em,text width=7em,inner sep=3em]                                       
\node[dff] (dff2) {D2};                             
\node[dff, right=2cm of dff2] (dff1) {D1};           
\node[dff, right=2cm of dff1] (dff0) {D0};        
%Connecting flip-flops together                    
\draw (dff2.out) -- ++(2,0) node[above]{};        
\draw (dff1.out) -- ++(2,0) node[above] {};         
\draw (dff0.out) -- ++(2,0) node[above]{};          
\draw(dff2.out) -| (2.3,1.5) node[above]{$Q2$};        
\draw(dff1.out) -|(6.8,1.2) node[above]{$Q1$};         
\draw(dff0.out) -|(12.4,2) node[above]{$Q0$};          
\draw(x.in 2) -|(-3,2)to[short] (12.4,2);              
\draw(x.in 1)-|(-2.5,1.5)to[short](2.3,1.5);          
\draw(-2,-2) node[above]{$Clk$} --(6,-2);            
\draw(6,-2) node[above]{} --(9.1,-2);                 
\draw(9.1,-2)--(9.1,-1.2) node[above]{};                 
\draw(8.9,-1.23)--(9.1,-1)--(9.3,-1.23);               
\draw(4.5,-2)--(4.5,-1.2) node[above]{};               
\draw(4.3,-1.23)--(4.5,-1)--(4.7,-1.23);            
\draw(0,-2)--(0,-1.2) node[above]{};               
\draw(-0.2,-1.23)--(0,-1)--(0.2,-1.23);
\end{tikzpicture}

\caption{mux}
\label{fig:1}
\end{figure}
\begin{enumerate}
\item $W = 0, X = 0, Y = 1, Z = 1$
\item $W = 1, X = 0, Y = 1, Z = 0$
\item $W = 0, X = 1, Y = 1, Z = 0$
\item $W = 1, X = 1, Y = 0, Z = 0$
\end{enumerate}
\section{Introduction}
	The above diagram is a 4:1 multiplexer where $W, X, Y, Z$ are the inputs of the multiplexer and $A$ is the output of the multiplexer.$R , S$ are the select lines of the multiplexer,which means:\newline
\begin{enumerate}
\item For $R = 0,S = 0$,the first input line $W$ is selected.
\item For $R = 0,S = 1$,the second input line $X$ is selected.
\item For $R = 1,S = 0$,the third input line $Y$ is selected.
\item For $R = 1,S = 1$,the fourth input line $Z$ is selected.
\end{enumerate}
Therefore,the resultant output expression of the multiplexer is $R'S'W + R'SX + RS'Y + RSZ$.
\section{Components}
\begin{table}[h]
	%%%%%%%%%%%%%%%%%%%%%%%%%%%%%%%%%%%%%%%%%%%%%%%%%%%%%%%%%%%%%%%%%%%%%%
%%                                                                  %%
%%  This is the header of a LaTeX2e file exported from Gnumeric.    %%
%%                                                                  %%
%%  This file can be compiled as it stands or included in another   %%
%%  LaTeX document. The table is based on the longtable package so  %%
%%  the longtable options (headers, footers...) can be set in the   %%
%%  preamble section below (see PRAMBLE).                           %%
%%                                                                  %%
%%  To include the file in another, the following two lines must be %%
%%  in the including file:                                          %%
%%        \def\inputGnumericTable{}                                 %%
%%  at the beginning of the file and:                               %%
%%        \input{name-of-this-file.tex}                             %%
%%  where the table is to be placed. Note also that the including   %%
%%  file must use the following packages for the table to be        %%
%%  rendered correctly:                                             %%
%%    \usepackage[latin1]{inputenc}                                 %%
%%    \usepackage{color}                                            %%
%%    \usepackage{array}                                            %%
%%    \usepackage{longtable}                                        %%
%%    \usepackage{calc}                                             %%
%%    \usepackage{multirow}                                         %%
%%    \usepackage{hhline}                                           %%
%%    \usepackage{ifthen}                                           %%
%%  optionally (for landscape tables embedded in another document): %%
%%    \usepackage{lscape}                                           %%
%%                                                                  %%
%%%%%%%%%%%%%%%%%%%%%%%%%%%%%%%%%%%%%%%%%%%%%%%%%%%%%%%%%%%%%%%%%%%%%%



%%  This section checks if we are begin input into another file or  %%
%%  the file will be compiled alone. First use a macro taken from   %%
%%  the TeXbook ex 7.7 (suggestion of Han-Wen Nienhuys).            %%
\def\ifundefined#1{\expandafter\ifx\csname#1\endcsname\relax}


%%  Check for the \def token for inputed files. If it is not        %%
%%  defined, the file will be processed as a standalone and the     %%
%%  preamble will be used.                                          %%
\ifundefined{inputGnumericTable}

%%  We must be able to close or not the document at the end.        %%
	\def\gnumericTableEnd{\end{document}}


%%%%%%%%%%%%%%%%%%%%%%%%%%%%%%%%%%%%%%%%%%%%%%%%%%%%%%%%%%%%%%%%%%%%%%
%%                                                                  %%
%%  This is the PREAMBLE. Change these values to get the right      %%
%%  paper size and other niceties.                                  %%
%%                                                                  %%
%%%%%%%%%%%%%%%%%%%%%%%%%%%%%%%%%%%%%%%%%%%%%%%%%%%%%%%%%%%%%%%%%%%%%%

	\documentclass[12pt%
			  %,landscape%
                    ]{report}
       \usepackage[latin1]{inputenc}
       \usepackage{fullpage}
       \usepackage{color}
       \usepackage{array}
       \usepackage{longtable}
       \usepackage{calc}
       \usepackage{multirow}
       \usepackage{hhline}
       \usepackage{ifthen}

	\begin{document}


%%  End of the preamble for the standalone. The next section is for %%
%%  documents which are included into other LaTeX2e files.          %%
\else

%%  We are not a stand alone document. For a regular table, we will %%
%%  have no preamble and only define the closing to mean nothing.   %%
    \def\gnumericTableEnd{}

%%  If we want landscape mode in an embedded document, comment out  %%
%%  the line above and uncomment the two below. The table will      %%
%%  begin on a new page and run in landscape mode.                  %%
%       \def\gnumericTableEnd{\end{landscape}}
%       \begin{landscape}


%%  End of the else clause for this file being \input.              %%
\fi

%%%%%%%%%%%%%%%%%%%%%%%%%%%%%%%%%%%%%%%%%%%%%%%%%%%%%%%%%%%%%%%%%%%%%%
%%                                                                  %%
%%  The rest is the gnumeric table, except for the closing          %%
%%  statement. Changes below will alter the table's appearance.     %%
%%                                                                  %%
%%%%%%%%%%%%%%%%%%%%%%%%%%%%%%%%%%%%%%%%%%%%%%%%%%%%%%%%%%%%%%%%%%%%%%

\providecommand{\gnumericmathit}[1]{#1} 
%%  Uncomment the next line if you would like your numbers to be in %%
%%  italics if they are italizised in the gnumeric table.           %%
%\renewcommand{\gnumericmathit}[1]{\mathit{#1}}
\providecommand{\gnumericPB}[1]%
{\let\gnumericTemp=\\#1\let\\=\gnumericTemp\hspace{0pt}}
 \ifundefined{gnumericTableWidthDefined}
        \newlength{\gnumericTableWidth}
        \newlength{\gnumericTableWidthComplete}
        \newlength{\gnumericMultiRowLength}
        \global\def\gnumericTableWidthDefined{}
 \fi
%% The following setting protects this code from babel shorthands.  %%
 \ifthenelse{\isundefined{\languageshorthands}}{}{\languageshorthands{english}}
%%  The default table format retains the relative column widths of  %%
%%  gnumeric. They can easily be changed to c, r or l. In that case %%
%%  you may want to comment out the next line and uncomment the one %%
%%  thereafter                                                      %%
\providecommand\gnumbox{\makebox[0pt]}
%%\providecommand\gnumbox[1][]{\makebox}

%% to adjust positions in multirow situations                       %%
\setlength{\bigstrutjot}{\jot}
\setlength{\extrarowheight}{\doublerulesep}

%%  The \setlongtables command keeps column widths the same across  %%
%%  pages. Simply comment out next line for varying column widths.  %%
\setlongtables

\setlength\gnumericTableWidth{%
	45pt+%
	30pt+%
	52pt+%
	60pt+%
0pt}
\def\gumericNumCols{4}
\setlength\gnumericTableWidthComplete{\gnumericTableWidth+%
         \tabcolsep*\gumericNumCols*2+\arrayrulewidth*\gumericNumCols}
\ifthenelse{\lengthtest{\gnumericTableWidthComplete > \linewidth}}%
         {\def\gnumericScale{\ratio{\linewidth-%
                        \tabcolsep*\gumericNumCols*2-%
                        \arrayrulewidth*\gumericNumCols}%
{\gnumericTableWidth}}}%
{\def\gnumericScale{1}}

%%%%%%%%%%%%%%%%%%%%%%%%%%%%%%%%%%%%%%%%%%%%%%%%%%%%%%%%%%%%%%%%%%%%%%
%%                                                                  %%
%% The following are the widths of the various columns. We are      %%
%% defining them here because then they are easier to change.       %%
%% Depending on the cell formats we may use them more than once.    %%
%%                                                                  %%
%%%%%%%%%%%%%%%%%%%%%%%%%%%%%%%%%%%%%%%%%%%%%%%%%%%%%%%%%%%%%%%%%%%%%%

\ifthenelse{\isundefined{\gnumericColA}}{\newlength{\gnumericColA}}{}\settowidth{\gnumericColA}{\begin{tabular}{@{}p{45pt*\gnumericScale}@{}}x\end{tabular}}
\ifthenelse{\isundefined{\gnumericColB}}{\newlength{\gnumericColB}}{}\settowidth{\gnumericColB}{\begin{tabular}{@{}p{30pt*\gnumericScale}@{}}x\end{tabular}}
\ifthenelse{\isundefined{\gnumericColC}}{\newlength{\gnumericColC}}{}\settowidth{\gnumericColC}{\begin{tabular}{@{}p{52pt*\gnumericScale}@{}}x\end{tabular}}
\ifthenelse{\isundefined{\gnumericColD}}{\newlength{\gnumericColD}}{}\settowidth{\gnumericColD}{\begin{tabular}{@{}p{60pt*\gnumericScale}@{}}x\end{tabular}}

\begin{tabular}[c]{%
	b{\gnumericColA}%
	b{\gnumericColB}%
	b{\gnumericColC}%
	b{\gnumericColD}%
	}

%%%%%%%%%%%%%%%%%%%%%%%%%%%%%%%%%%%%%%%%%%%%%%%%%%%%%%%%%%%%%%%%%%%%%%
%%  The longtable options. (Caption, headers... see Goosens, p.124) %%
%	\caption{The Table Caption.}             \\	%
% \hline	% Across the top of the table.
%%  The rest of these options are table rows which are placed on    %%
%%  the first, last or every page. Use \multicolumn if you want.    %%

%%  Header for the first page.                                      %%
%	\multicolumn{4}{c}{The First Header} \\ \hline 
%	\multicolumn{1}{c}{colTag}	%Column 1
%	&\multicolumn{1}{c}{colTag}	%Column 2
%	&\multicolumn{1}{c}{colTag}	%Column 3
%	&\multicolumn{1}{c}{colTag}	\\ \hline %Last column
%	\endfirsthead

%%  The running header definition.                                  %%
%	\hline
%	\multicolumn{4}{l}{\ldots\small\slshape continued} \\ \hline
%	\multicolumn{1}{c}{colTag}	%Column 1
%	&\multicolumn{1}{c}{colTag}	%Column 2
%	&\multicolumn{1}{c}{colTag}	%Column 3
%	&\multicolumn{1}{c}{colTag}	\\ \hline %Last column
%	\endhead

%%  The running footer definition.                                  %%
%	\hline
%	\multicolumn{4}{r}{\small\slshape continued\ldots} \\
%	\endfoot

%%  The ending footer definition.                                   %%
%	\multicolumn{4}{c}{That's all folks} \\ \hline 
%	\endlastfoot
%%%%%%%%%%%%%%%%%%%%%%%%%%%%%%%%%%%%%%%%%%%%%%%%%%%%%%%%%%%%%%%%%%%%%%

\hhline{|-|-|-|-}
	 \multicolumn{1}{|p{\gnumericColA}|}%
	{\gnumericPB{\raggedright}\textbf{ESP32}}
	&\multicolumn{1}{p{\gnumericColB}|}%
	{\gnumericPB{\raggedright}\textbf{LCD Pins}}
	&\multicolumn{1}{p{\gnumericColC}|}%
	{\gnumericPB{\raggedright}\textbf{LCD Pin Label}}
	&\multicolumn{1}{p{\gnumericColD}|}%
	{\gnumericPB{\raggedright}\textbf{LCD Pin Description}}
\\
\hhline{|----|}
	 \multicolumn{1}{|p{\gnumericColA}|}%
	{\gnumericPB{\raggedright}GND}
	&\multicolumn{1}{p{\gnumericColB}|}%
	{\gnumericPB{\raggedright}1}
	&\multicolumn{1}{p{\gnumericColC}|}%
	{\gnumericPB{\raggedright}GND }
	&\multicolumn{1}{p{\gnumericColD}|}%
	{\setlength{\gnumericMultiRowLength}{0pt}%
	 \addtolength{\gnumericMultiRowLength}{\gnumericColD}%
	 \multirow{2}[1]{\gnumericMultiRowLength}{%
	 }}
\\
\hhline{|---|~}
	 \multicolumn{1}{|p{\gnumericColA}|}%
	{\gnumericPB{\raggedright}5V}
	&\multicolumn{1}{p{\gnumericColB}|}%
	{\gnumericPB{\raggedright}2}
	&\multicolumn{1}{p{\gnumericColC}|}%
	{\gnumericPB{\raggedright}Vcc}
	&\multicolumn{1}{p{\gnumericColD}|}%
	{}
\\
\hhline{|----|}
	 \multicolumn{1}{|p{\gnumericColA}|}%
	{\gnumericPB{\raggedright}GND}
	&\multicolumn{1}{p{\gnumericColB}|}%
	{\gnumericPB{\raggedright}3}
	&\multicolumn{1}{p{\gnumericColC}|}%
	{\gnumericPB{\raggedright}Vee}
	&\multicolumn{1}{p{\gnumericColD}|}%
	{\gnumericPB{\raggedright}Contrast}
\\
\hhline{|----|}
	 \multicolumn{1}{|p{\gnumericColA}|}%
	{\gnumericPB{\raggedright}GPIO 19}
	&\multicolumn{1}{p{\gnumericColB}|}%
	{\gnumericPB{\raggedright}4}
	&\multicolumn{1}{p{\gnumericColC}|}%
	{\gnumericPB{\raggedright}RS}
	&\multicolumn{1}{p{\gnumericColD}|}%
	{\gnumericPB{\raggedright}Register Select}
\\
\hhline{|----|}
	 \multicolumn{1}{|p{\gnumericColA}|}%
	{\gnumericPB{\raggedright}GND}
	&\multicolumn{1}{p{\gnumericColB}|}%
	{\gnumericPB{\raggedright}5}
	&\multicolumn{1}{p{\gnumericColC}|}%
	{\gnumericPB{\raggedright}R/W}
	&\multicolumn{1}{p{\gnumericColD}|}%
	{\gnumericPB{\raggedright}Read/Write}
\\
\hhline{|----|}
	 \multicolumn{1}{|p{\gnumericColA}|}%
	{\gnumericPB{\raggedright}GPIO 23}
	&\multicolumn{1}{p{\gnumericColB}|}%
	{\gnumericPB{\raggedright}6}
	&\multicolumn{1}{p{\gnumericColC}|}%
	{\gnumericPB{\raggedright}EN}
	&\multicolumn{1}{p{\gnumericColD}|}%
	{\gnumericPB{\raggedright}Enable}
\\
\hhline{|----|}
	 \multicolumn{1}{|p{\gnumericColA}|}%
	{\gnumericPB{\raggedright}GPIO 18}
	&\multicolumn{1}{p{\gnumericColB}|}%
	{\gnumericPB{\raggedright}11}
	&\multicolumn{1}{p{\gnumericColC}|}%
	{\gnumericPB{\raggedright}DB4}
	&\multicolumn{1}{p{\gnumericColD}|}%
	{\gnumericPB{\raggedright}Serial Connection}
\\
\hhline{|----|}
	 \multicolumn{1}{|p{\gnumericColA}|}%
	{\gnumericPB{\raggedright}GPIO 17}
	&\multicolumn{1}{p{\gnumericColB}|}%
	{\gnumericPB{\raggedright}12}
	&\multicolumn{1}{p{\gnumericColC}|}%
	{\gnumericPB{\raggedright}DB5}
	&\multicolumn{1}{p{\gnumericColD}|}%
	{\gnumericPB{\raggedright}Serial Connection}
\\
\hhline{|----|}
	 \multicolumn{1}{|p{\gnumericColA}|}%
	{\gnumericPB{\raggedright}GPIO 16}
	&\multicolumn{1}{p{\gnumericColB}|}%
	{\gnumericPB{\raggedright}13}
	&\multicolumn{1}{p{\gnumericColC}|}%
	{\gnumericPB{\raggedright}DB6}
	&\multicolumn{1}{p{\gnumericColD}|}%
	{\gnumericPB{\raggedright}Serial Connection}
\\
\hhline{|----|}
	 \multicolumn{1}{|p{\gnumericColA}|}%
	{\gnumericPB{\raggedright}GPIO 15}
	&\multicolumn{1}{p{\gnumericColB}|}%
	{\gnumericPB{\raggedright}14}
	&\multicolumn{1}{p{\gnumericColC}|}%
	{\gnumericPB{\raggedright}DB7}
	&\multicolumn{1}{p{\gnumericColD}|}%
	{\gnumericPB{\raggedright}Serial Connection}
\\
\hhline{|----|}
	 \multicolumn{1}{|p{\gnumericColA}|}%
	{\gnumericPB{\raggedright}5V}
	&\multicolumn{1}{p{\gnumericColB}|}%
	{\gnumericPB{\raggedright}15}
	&\multicolumn{1}{p{\gnumericColC}|}%
	{\gnumericPB{\raggedright}LED+}
	&\multicolumn{1}{p{\gnumericColD}|}%
	{\gnumericPB{\raggedright}Backlight}
\\
\hhline{|----|}
	 \multicolumn{1}{|p{\gnumericColA}|}%
	{\gnumericPB{\raggedright}GND}
	&\multicolumn{1}{p{\gnumericColB}|}%
	{\gnumericPB{\raggedright}16}
	&\multicolumn{1}{p{\gnumericColC}|}%
	{\gnumericPB{\raggedright}LED-}
	&\multicolumn{1}{p{\gnumericColD}|}%
	{\gnumericPB{\raggedright}Backlight}
\\
\hhline{|-|-|-|-|}
\end{tabular}

\ifthenelse{\isundefined{\languageshorthands}}{}{\languageshorthands{\languagename}}
\gnumericTableEnd

\caption{contents}
\label{table 1}
\end{table}
	\pagebreak
\section{Hardware}
	\begin{enumerate}
\item Connect the COM of the seven-segment display to 5V and dot of the seven-segment to the ground.
\item Now connect any one of the pin of the seven-segment to pin no.2(digital).
\item Pin no.s 5,6,7,8 of the arduino should be initially connected to ground.
\item Now move pin no.s 5,6,7,8 accordingly and for the right combination the second pin of the arduino becomes high and the seven segement display glows.
\end{enumerate}
\begin{table}[h]
\begin{center}
	\captionof{table}{Table2}
\label{table:2}
\begin{tabular}{|p{3cm}|p{1cm}|p{1cm}|p{1cm}|p{1cm}|p{1cm}|p{1cm}|p{1cm}|}                                           
	\hline                                                
	\multicolumn{8}{|c|}{7447 - Display}\\                                                                    
	\hline                                                
	7447& $\bar{a}$ & $\bar{b}$ & $\bar{c}$ & $\bar{d}$ & $\bar{e}$ & $\bar{f}$ & $\bar{g}$\\                                                                    
	\hline                                                
	Display& a& b& c& d& e& f& g\\                                                                            
	\hline                                        
\end{tabular}

\end{center}
\caption{truth table}
\label{table 2}
\end{table}
The K-map for this truth table will be a two variable K-map and it will be as follows:
\begin{figure}[h]
		\begin{center}
	\begin{karnaugh-map}[2][2][1][$R$][$S$]
		\minterms{1,2}
		\autoterms[0]
	\end{karnaugh-map}
	\end{center}	

\caption{k-map}
\label{fig2}
\end{figure}

So,the resultant expression of A is $A = R'S + RS'$.
\pagebreak
\section{Software}

The code below can help in solving the above problem.
\lstinputlisting{q19.cpp}
\end{document}
